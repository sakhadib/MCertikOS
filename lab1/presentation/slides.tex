\documentclass{beamer}

% Theme choice (optional)
\usetheme{Singapore}

% Background colors
%\definecolor{primaryBG}{RGB}{22,66,60} % Dark green for main background
%\definecolor{secondaryBG}{RGB}{106,156,137} % Medium green for title and headers
%\definecolor{slideBG}{RGB}{233,239,236} % Light color for slide background

% Foreground colors
%\definecolor{primaryText}{RGB}{24,28,20} % Dark greenish-black for main text
%\definecolor{secondaryText}{RGB}{60,61,55} % Dark grayish for secondary text
%\definecolor{highlightText}{RGB}{105,117,101} % Muted green for highlights
%\definecolor{titleText}{RGB}{236,223,204} % Light beige for title and header text

% Apply background colors
%\setbeamercolor{background canvas}{bg=slideBG} % Light background for slides
%\setbeamercolor{frametitle}{bg=secondaryBG, fg=titleText} % Title background with light title text
%\setbeamercolor{section in head/foot}{bg=secondaryBG, fg=titleText} % Section header background

% Apply foreground colors
%\setbeamercolor{normal text}{fg=primaryText} % Main text color
%\setbeamercolor{author in head/foot}{bg=primaryBG, fg=titleText} % Author in footer
%\setbeamercolor{title in head/foot}{bg=primaryBG, fg=titleText} % Title in footer
%\setbeamercolor{date in head/foot}{bg=primaryBG, fg=titleText} % Date in footer

% Package for including images
\usepackage{graphicx}
\usepackage{listings}
\usepackage{xcolor}

%for code styling ---------------------------------------
\definecolor{mBlue}{rgb}{0,0.4,0.7}
\definecolor{mGray}{rgb}{0.5,0.5,0.5}
\definecolor{mPurple}{rgb}{0.58,0,0.82}
% \definecolor{backgroundColour}{HTML}{e6f0ee}
\definecolor{backgroundColour}{HTML}{ffffff}
\definecolor{mAdib}{rgb}{0,0.8,0}


\lstdefinestyle{code}{
	backgroundcolor=\color{backgroundColour},   
	commentstyle=\color{mGray},
	keywordstyle=\color{mBlue}\bfseries,
	numberstyle=\color{mGray},
	stringstyle=\color{mPurple},
	basicstyle=\ttfamily\footnotesize,
	breakatwhitespace=false,         
	breaklines=true,                 
	captionpos=b,
	keepspaces=true,                                            
	morekeywords={Input, Output,},
	showspaces=false,                
	showstringspaces=false,
	xleftmargin=5pt,
	numbers=left,
	numbersep=12pt,
	showtabs=false,                  
	tabsize=4,
	% frame=single,
	language=c
}


% Title, author, and other details
\title{\textbf{Operating Systems Lab}}
\subtitle{Lab - 1: Bootloader \& Physical Memory Management}

\author{\textbf{Adib Sakhawat} \& \textbf{Yasir Raiyan}}
\institute[ ]{ID: 210042106 \& 210042152\\Software Engineering\\Dept. of Computer Science and Engineering\\Islamic University of Technology}
\date{\today}

% Custom footer setup
\setbeamertemplate{footline}{
	\leavevmode%
	\hbox{%
		\begin{beamercolorbox}[wd=.1\paperwidth,ht=2.5ex,dp=1.125ex]{author in head/foot}%
			% Blank space on the left
		\end{beamercolorbox}%
		
		\begin{beamercolorbox}[wd=.3\paperwidth,ht=2.5ex,dp=1.125ex,left]{author in head/foot}%
			\usebeamerfont{author in head/foot} \insertshortauthor % Name
		\end{beamercolorbox}%
		
		\begin{beamercolorbox}[wd=.3\paperwidth,ht=2.5ex,dp=1.125ex,left]{author in head/foot}%
			\usebeamerfont{title in head/foot} \insertshorttitle % Name
		\end{beamercolorbox}%

		
		\begin{beamercolorbox}[wd=.2\paperwidth,ht=2.5ex,dp=1.125ex,right]{date in head/foot}%
			\usebeamerfont{date in head/foot} \insertframenumber/\inserttotalframenumber % Page number (right-aligned)
		\end{beamercolorbox}%
		
		\begin{beamercolorbox}[wd=.1\paperwidth,ht=2.5ex,dp=1.125ex]{author in head/foot}%
			% Blank space on the right
		\end{beamercolorbox}%
	}%
	\vskip0pt%
}

\begin{document}
	
	% Title slide
	\begin{frame}
		\titlepage
	\end{frame}
	

	% Slide 1: Introduction
	\section{Bootloader}
	\begin{frame}
		\frametitle{What is a Bootloader?}
		\begin{itemize}
			\item A small program that runs when a computer starts.
			\item Loads the operating system into memory.
			\item Begins the execution of the OS.
			\item Essential for the OS to function.
		\end{itemize}
	\end{frame}
	
	\section{Physical Memory Management}		
	% Slide 1: Introduction to Physical Memory Management
	\begin{frame}
		\frametitle{Introduction to Physical Memory Management}
		\begin{itemize}
			\item \textbf{Physical Memory:} The hardware memory (RAM) used by programs.
			\item \textbf{Role of OS:} Manages memory allocation and ensures efficient use.
			\item \textbf{Allocation:} Decides how memory is divided and organized.
		\end{itemize}
	\end{frame}
	
	% Slide 2: Memory Allocation
	\begin{frame}
		\frametitle{Memory Allocation}
		\begin{itemize}
			\item \textbf{Fixed Partitioning:} Predefined memory blocks, simple but inefficient.
			\item \textbf{Dynamic Partitioning:} Flexible memory allocation based on process size.
			\item \textbf{Contiguous Allocation:} Memory blocks assigned sequentially.
			\item \textbf{Non-contiguous Allocation:} Allows memory blocks scattered across RAM.
		\end{itemize}
	\end{frame}
	
	% Slide 3: Paging
	\begin{frame}
		\frametitle{Paging}
		\begin{itemize}
			\item \textbf{Pages:} Memory divided into fixed-size pages.
			\item \textbf{Page Table:} Maps logical pages to physical frames.
			\item \textbf{Page Frames:} Fixed-size blocks in physical memory.
		\end{itemize}
	\end{frame}
	
	% Slide 4: Segmentation
	\begin{frame}
		\frametitle{Segmentation}
		\begin{itemize}
			\item \textbf{Segments:} Memory divided based on logical divisions (code, data, stack).
			\item \textbf{Segment Table:} Maps segments to physical addresses.
			\item \textbf{Logical vs. Physical Address:} Logical address is used by the program, physical address by hardware.
		\end{itemize}
	\end{frame}
	
	% Slide 5: Memory Protection and Fragmentation
	\begin{frame}
		\frametitle{Memory Protection and Fragmentation}
		\begin{itemize}
			\item \textbf{Memory Protection:} Prevents one process from accessing another's memory.
			\item \textbf{Fragmentation:} 
			\begin{itemize}
				\item \textbf{Internal Fragmentation:} Unused memory within allocated space.
				\item \textbf{External Fragmentation:} Free memory scattered across.
			\end{itemize}
			\item \textbf{Solutions:} Compaction, Paging, and Segmentation.
		\end{itemize}
	\end{frame}
	
		
		
	\section{MATIntro Layer}
	\begin{frame}
		\frametitle{MATIntro Layer: Memory Management Overview}
		
		\begin{itemize}
			\item The code defines a Memory Allocation Table (MAT) to manage physical memory pages.
			\item Pages are represented as 4KB units, and permissions are assigned to each page.
		\end{itemize}
			
		
	\end{frame}
	
	
	\begin{frame}
		\frametitle{Key Components:}
		
		\begin{itemize}
			\item \texttt{NUM\_PAGES}: Number of physical pages available in the system.
			\item \texttt{struct ATStruct}: Represents each page with permission and allocation status.
			\item \texttt{AT[1 << 20]}: Array storing information for each physical page (up to 4GB memory)
		\end{itemize}
	\end{frame}
	
	\begin{frame}
		\frametitle{Core Functions:}
		
		\begin{itemize}
			\item \texttt{get\_nps(), set\_nps()}: Get/set number of available pages.
			\item \texttt{at\_is\_norm()}: Checks if a page has normal permission.
			\item \texttt{at\_set\_perm()}: Sets page permission and marks it as unallocated.
			\item \texttt{at\_is\_allocated()}: Checks if a page is allocated.
			\item \texttt{at\_set\_allocated()}: Sets allocation status of a page.
		\end{itemize}
		
	\end{frame}
	
	
	\section{MATInit Layer}
	
	% Slide 1: Introduction to pmem_init()
	\begin{frame}
		\frametitle{Introduction to \texttt{pmem\_init()}}
		\begin{itemize}
			\item Initializes physical memory and allocation table (AT).
			\item Configures permissions for memory pages based on the memory map.
			\item Pages are 4KB in size.
			\item \texttt{VM\_USERLO/VM\_USERHI}: Define user-space memory boundaries.
		\end{itemize}
	\end{frame}
	
	% Slide 2: Calculating Physical Memory Pages
	\begin{frame}
		\frametitle{Calculating Physical Memory Pages}
		\begin{itemize}
			\item \texttt{nps}: Total number of physical pages.
			\item Pages calculated as: \texttt{nps = (highestAddr + 1) / PAGESIZE}.
			\item Fetch memory map rows with \texttt{get\_size()}.
			\item Determine highest address using \texttt{get\_mms()} and \texttt{get\_mml()}.
		\end{itemize}
	\end{frame}
	
	% Slide 3: Initializing the Physical Allocation Table
	\begin{frame}
		\frametitle{Initializing the Physical Allocation Table}
		\begin{itemize}
			\item \textbf{Kernel-reserved addresses:}
			\begin{itemize}
				\item Pages < \texttt{VM\_USERLO\_PI} or >= \texttt{VM\_USERHI\_PI} are reserved.
				\item Set permission to 1 for these pages.
			\end{itemize}
			\item \textbf{User-space pages:}
			\begin{itemize}
				\item Pages within [\texttt{VM\_USERLO}, \texttt{VM\_USERHI}] can be used if marked available.
				\item Permissions are based on memory map.
			\end{itemize}
		\end{itemize}
	\end{frame}
	
	% Slide 4: Kernel-Reserved Pages
	\begin{frame}
		\frametitle{Kernel-Reserved Pages}
		\begin{itemize}
			\item $Pages < \texttt{VM\_USERLO\_PI}$ and $Pages >= \texttt{VM\_USERHI\_PI}$ are reserved.
			\item Set permission to 1.
		\end{itemize}
		
	\end{frame}
	
	% Slide 5: User-Space Page Initialization
	\begin{frame}
		\frametitle{User-Space Page Initialization}
		\begin{itemize}
			\item Pages within [\texttt{VM\_USERLO}, \texttt{VM\_USERHI}] are checked.
			\item Permissions set based on memory map.
			\item Pages are marked as:
			\begin{itemize}
				\item \textbf{2}: Usable.
				\item \textbf{0}: Unavailable (partial pages considered unavailable).
			\end{itemize}
		\end{itemize}
	\end{frame}
	
	
	% Slide 6: Final Page Permission Setup
	\begin{frame}
		\frametitle{Final Page Permission Setup}
		\begin{itemize}
			\item Loop through the memory map.
			\item Set permission based on usability:
			\begin{itemize}
				\item \texttt{2}: Usable pages.
				\item \texttt{0}: Unavailable or partially usable pages.
			\end{itemize}
		\end{itemize}
		
		
	\end{frame}
	
	
	
	\section{MATOp Layer}
	\begin{frame}
		\frametitle{Introduction to Page Allocation}
		\begin{itemize}
			\item \textbf{Page Allocation}: Managing physical memory by allocating and freeing pages.
			\item \textbf{Key Functions}:
			\begin{itemize}
				\item \texttt{palloc()}: Allocates a physical page.
				\item \texttt{pfree()}: Frees an allocated page.
			\end{itemize}
		\end{itemize}
	\end{frame}
	
	\begin{frame}
		\frametitle{Understanding Physical Pages}
		\begin{itemize}
			\item \textbf{Physical Page Size}: Defined as 4KB (\texttt{PAGESIZE = 4096}).
			\item \textbf{User Space Limits}:
			\begin{itemize}
				\item \texttt{VM\_USERLO}: Start of user-space memory (0x40000000).
				\item \texttt{VM\_USERHI}: End of user-space memory (0xF0000000).
			\end{itemize}
			\item \textbf{Page Index Range}: 
			\begin{itemize}
				\item \texttt{VM\_USERLO\_PI} to \texttt{VM\_USERHI\_PI} determines valid page indices.
			\end{itemize}
		\end{itemize}
	\end{frame}
	
	\begin{frame}
		\frametitle{Overview of \texttt{palloc()}}
		\begin{itemize}
			\item \textbf{Purpose}: Allocate a physical page.
			\item \textbf{Process}:
			\begin{enumerate}
				\item \textbf{Check Availability}: Ensure pages are available in the allocation table (AT).
				\item \textbf{Scan for Unallocated Pages}: Look for the first unallocated page with normal permissions.
				\item \textbf{Mark as Allocated}: If found, mark the page and return its index.
			\end{enumerate}
		\end{itemize}
	\end{frame}
	
	\begin{frame}
		\frametitle{Initialization and Scanning}
		\begin{itemize}
			\item \textbf{Starting Point}: The allocation starts from the variable \texttt{next}, initialized to \texttt{VM\_USERLO\_PI}.
			\item \textbf{Loop Logic}:
			\begin{itemize}
				\item Scan from \texttt{next} to \texttt{VM\_USERHI\_PI}.
				\item Wrap around to \texttt{VM\_USERLO\_PI} if the end is reached.
			\end{itemize}
			\item \textbf{Return Value}: Returns the index of the allocated page or 0 if none are available.
		\end{itemize}
	\end{frame}
	
	\begin{frame}
		\frametitle{Optimizing with Memoization}
		\begin{itemize}
			\item \textbf{Memoization Concept}: Store the last allocated page to avoid scanning the entire AT repeatedly.
			\item \textbf{Efficiency}: Reduces overhead by starting the scan from the last allocated page.
		\end{itemize}
	\end{frame}
	
	\begin{frame}
		\frametitle{Overview of \texttt{pfree()}}
		\begin{itemize}
			\item \textbf{Purpose}: Free a physical page.
			\item \textbf{Process}:
			\begin{itemize}
				\item Takes an index (\texttt{pfree\_index}) of the page to be freed.
				\item Calls \texttt{at\_set\_allocated(pfree\_index, 0)} to mark the page as unallocated.
			\end{itemize}
		\end{itemize}
	\end{frame}
	
	\section{Conclusion}
	\begin{frame}
		\frametitle{Conclusion}
		\begin{itemize}
			\item Efficient memory management is crucial for the performance and stability of operating systems.
			\item Understanding the role of bootloaders and memory allocation techniques is essential.
			\item Key functions such as \texttt{palloc()} and \texttt{pfree()} play a vital role in managing physical memory.
			\item Ongoing optimization techniques can enhance memory allocation efficiency and system performance.
		\end{itemize}
	\end{frame}
	
	\begin{frame}
		\frametitle{Thank You!}
		\begin{center}
			\textbf{Thank you for your attention!} \\
			\vspace{1em}
		\end{center}
	\end{frame}
	
	
	
	
	

	
	
	
	

	
\end{document}
